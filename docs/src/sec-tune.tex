%=========================================================================
% sec-tune
%=========================================================================

\section{Tuning the Shallow Water Simulation}
\label{sec-tune}

The following optimizations are explored in this section:

\begin{itemize}
  \item Batching time-steps with ghost cells
  \item Vectorizing with AVX extensions
  \item Eliminating redundant computation
  \item Offloading vs. natively executing the parallel kernel
  \item Overlapping computation with communication
  \item Implementing alternative physics modeling algorithms
\end{itemize}

\subsection{Batching Time-Steps with Ghost Cells}
\label{sec-tune-batching}

\subsection{Vectorizing with AVX Extensions}
\label{sec-tune-vectorizing}

\noindent Vectorizing with AVX Extensions can fall under two general categories:

\begin{enumerate}[1.]
    \item Having the compiler auto-vectorize your code for you, and
    \item Writing your own vectorized kernels for use as subroutines.
\end{enumerate}

Both approaches have benefits and detriments.  For example, when using the compiler to vectorize your code you save yourself the pain of having to reason about which register has what data, what SSE/AVX function callbacks wrap the appropriate instruction level code, etc.  This provides great convenience, but does come at a cost.  The compiler is only as smart as you let it be, and if you arrange your code improperly, you may not only slow down your program but you may very well yield the wrong result by accident.  Writing your own vectorized code, on the other hand, can be quite difficult to get right.  Often times it may be best to take a mixed approach, where you examine the instructions the compiler generated for a given code segment and determine whether or not you can refine this approach further.

\subsubsection{Auto-vectorization Using ICC}
\label{sec-tune-vectorizing-auto-vectorizing}

As discussed previously, you can use the \texttt{amplxe} tool to help with profiling, as well as use the intel compiler to generate an \texttt{optrpt} file describing what was / was not vectorized, how effective it was if vectorization occurred, and why vectorization did not occur if that were the case.  On the note of when vectorization does not occur, we would like to mention a couple of things:

\begin{enumerate}[1.]
    \item Not all loops are created equal

    That is not every loop can be vectorized, either because its length cannot be known at compile-time, or because the data elements being accessed cannot be made to execute in parallel.

    \item Understanding the codes in the \texttt{.optrpt} file can help you rearrange your code so that it is vectorized (recalling from 1 that not all of this is possible).

    We found the following presentation extremely useful in rearranging loop statements:\\\\ \centerline{https://engineering.purdue.edu/~milind/ece573/2011spring/lecture-14.pdf}

    \item Introducing compile-time constants / \texttt{constexpr} members of a class can go a long way in assisting the compiler understand what can / cannot be vectorized.

    \item Taking great care to enforce memory alignment as well as declare said alignment to the compiler will also enable it to vectorize even more.
\end{enumerate}

We leave items 1 and 2 as an exercise for the reader ;)  For item 3, we must first acknowledge that the purpose of the data type \texttt{std::array} is largely just for compilation hints, and we can wield this to our advantage.  In the original implementation of \texttt{Shallow2d.h}, we had that

{\tiny
\begin{lstlisting}
    // Type parameters for solver
    typedef float real;
    typedef std::array<real,3> vec;
\end{lstlisting}
}

were the primary solver types used throughout the program.  The issue, though, is that regardless of us declaring with \texttt{<real,3>}, with high probability (given the architectures we are compiling on) this will get padded to 16 bytes regardless.  The issue with this padding though, is that it is not guaranteed to be a \texttt{float} and treating it as such can potentially give problems.  Recognizing this, we can modify this definition to be:

{\tiny
\begin{lstlisting}
    // Type parameters for solver
    #define VEC_DIM 4 // change this and we all die...
    typedef float real;
    #ifdef __INTEL_COMPILER
        typedef __declspec(align(16)) std::array<real, VEC_DIM> vec;
    #else // GCC
        typedef __attribute__ ((aligned(16))) std::array<real, VEC_DIM> vec;
    #endif

    // allow loop unrolling over `vec`
    static constexpr int vec_size  = VEC_DIM;
    static constexpr int VEC_ALIGN = 16;
\end{lstlisting}
}

Woah.  Ugly.  But necessary.  What this does is ensure that we will have 4 floats per type vec, and depending on the compiler you are using also declares the alignment of this type.  This stage may not be necessary.  We also define two \texttt{static constexpr} members to allow us to use say \texttt{Physics::vec\_size} as a compile time constant to allow us to unroll loops / \texttt{\_\_assume\_aligned} on memory.  For example, the original computation of the corrector step in \texttt{Central2d.h} was:
{\tiny
\begin{lstlisting}
    // Corrector (finish the step)
    for (int iy = nghost-io; iy < ny+nghost-io; ++iy) {
        for (int ix = nghost-io; ix < nx+nghost-io; ++ix) {
            for (int m = 0; m < v(ix,iy).size(); ++m) {
                v(ix,iy)[m] =
                    0.2500 * ( u(ix,  iy)[m] + u(ix+1,iy  )[m] +
                               u(ix,iy+1)[m] + u(ix+1,iy+1)[m] ) -
                    0.0625 * ( ux(ix+1,iy  )[m] - ux(ix,iy  )[m] +
                               ux(ix+1,iy+1)[m] - ux(ix,iy+1)[m] +
                               uy(ix,  iy+1)[m] - uy(ix,  iy)[m] +
                               uy(ix+1,iy+1)[m] - uy(ix+1,iy)[m] ) -
                    dtcdx2 * ( f(ix+1,iy  )[m] - f(ix,iy  )[m] +
                               f(ix+1,iy+1)[m] - f(ix,iy+1)[m] ) -
                    dtcdy2 * ( g(ix,  iy+1)[m] - g(ix,  iy)[m] +
                               g(ix+1,iy+1)[m] - g(ix+1,iy)[m] );
            }
        }
    }
\end{lstlisting}
}
Though extremely verbose, we can now use these new additions to write a new loop:

{\tiny
\begin{lstlisting}
    // Corrector (finish the step)
    for (int iy = nghost-io; iy < ny+nghost-io; ++iy) {
        for (int ix = nghost-io; ix < nx+nghost-io; ++ix) {
            /* Nomenclature:
             *     u_x0_y0 <- u(ix  , iy  )
             *     u_x1_y0 <- u(ix+1, iy  )
             *     u_x0_y1 <- u(ix  , iy+1)
             *     u_x1_y1 <- u(ix+1, iy+1)
             */
            // The final result
            real *v_ix_iy = v(ix, iy).data();       __assume_aligned(v_ix_iy, Physics::VEC_ALIGN);

            // grab u
            real *u_x1_y0 = u(ix+1, iy  ).data();   __assume_aligned(u_x1_y0, Physics::VEC_ALIGN);
            real *u_x0_y0 = u(ix  , iy  ).data();   __assume_aligned(u_x0_y0, Physics::VEC_ALIGN);
            real *u_x0_y1 = u(ix  , iy+1).data();   __assume_aligned(u_x0_y1, Physics::VEC_ALIGN);
            real *u_x1_y1 = u(ix+1, iy+1).data();   __assume_aligned(u_x1_y1, Physics::VEC_ALIGN);

            // grab ux
            real *ux_x0_y0 = ux(ix  , iy  ).data(); __assume_aligned(ux_x0_y0, Physics::VEC_ALIGN);
            real *ux_x1_y0 = ux(ix+1, iy  ).data(); __assume_aligned(ux_x1_y0, Physics::VEC_ALIGN);
            real *ux_x0_y1 = ux(ix  , iy+1).data(); __assume_aligned(ux_x0_y1, Physics::VEC_ALIGN);
            real *ux_x1_y1 = ux(ix+1, iy+1).data(); __assume_aligned(ux_x1_y1, Physics::VEC_ALIGN);

            // grab uy
            real *uy_x0_y0 = uy(ix  , iy  ).data(); __assume_aligned(uy_x0_y0, Physics::VEC_ALIGN);
            real *uy_x1_y0 = uy(ix+1, iy  ).data(); __assume_aligned(uy_x1_y0, Physics::VEC_ALIGN);
            real *uy_x0_y1 = uy(ix  , iy+1).data(); __assume_aligned(uy_x0_y1, Physics::VEC_ALIGN);
            real *uy_x1_y1 = uy(ix+1, iy+1).data(); __assume_aligned(uy_x1_y1, Physics::VEC_ALIGN);

            // grab f
            real *f_x0_y0 = f(ix  , iy  ).data();   __assume_aligned(f_x0_y0, Physics::VEC_ALIGN);
            real *f_x1_y0 = f(ix+1, iy  ).data();   __assume_aligned(f_x1_y0, Physics::VEC_ALIGN);
            real *f_x0_y1 = f(ix  , iy+1).data();   __assume_aligned(f_x0_y1, Physics::VEC_ALIGN);
            real *f_x1_y1 = f(ix+1, iy+1).data();   __assume_aligned(f_x1_y1, Physics::VEC_ALIGN);

            // grab g
            real *g_x0_y0 = g(ix  , iy  ).data();   __assume_aligned(g_x0_y0, Physics::VEC_ALIGN);
            real *g_x1_y0 = g(ix+1, iy  ).data();   __assume_aligned(g_x1_y0, Physics::VEC_ALIGN);
            real *g_x0_y1 = g(ix  , iy+1).data();   __assume_aligned(g_x0_y1, Physics::VEC_ALIGN);
            real *g_x1_y1 = g(ix+1, iy+1).data();   __assume_aligned(g_x1_y1, Physics::VEC_ALIGN);

            #pragma simd
            for(int m = 0; m < Physics::vec_size; ++m) {
                v_ix_iy[m] =
                    0.2500f * ( u_x0_y0[m]  + u_x1_y0[m]    +
                                u_x0_y1[m]  + u_x1_y1[m]  ) -
                    0.0625f * ( ux_x1_y0[m] - ux_x0_y0[m]   +
                                ux_x1_y1[m] - ux_x0_y1[m]   +
                                uy_x0_y1[m] - uy_x0_y0[m]   +
                                uy_x1_y1[m] - uy_x1_y0[m] ) -
                    dtcdx2  * ( f_x1_y0[m]  - f_x0_y0[m]    +
                                f_x1_y1[m]  - f_x0_y1[m]  ) -
                    dtcdy2  * ( g_x0_y1[m]  - g_x0_y0[m]    +
                                g_x1_y1[m]  - g_x1_y0[m]  );                    
            }
        }
    }
\end{lstlisting}
}

The reasoning is that these traits now enable the intel compiler to do what it does best: vectorize like there ain't no tomorrow:

{\tiny
\begin{lstlisting}
LOOP BEGIN at central2d.h(368,5)
remark #15542: loop was not vectorized: inner loop was already vectorized

   LOOP BEGIN at central2d.h(369,9)
      remark #15542: loop was not vectorized: inner loop was already vectorized

      LOOP BEGIN at central2d.h(410,13)
         remark #15388: vectorization support: reference F32 has aligned access   [ central2d.h(411,17) ]
         remark #15388: vectorization support: reference F32 has aligned access   [ central2d.h(411,17) ]
         remark #15388: vectorization support: reference F32 has aligned access   [ central2d.h(411,17) ]
         remark #15388: vectorization support: reference F32 has aligned access   [ central2d.h(411,17) ]
         remark #15388: vectorization support: reference F32 has aligned access   [ central2d.h(411,17) ]
         remark #15388: vectorization support: reference F32 has aligned access   [ central2d.h(411,17) ]
         remark #15388: vectorization support: reference F32 has aligned access   [ central2d.h(411,17) ]
         remark #15388: vectorization support: reference F32 has aligned access   [ central2d.h(411,17) ]
         remark #15388: vectorization support: reference F32 has aligned access   [ central2d.h(411,17) ]
         remark #15388: vectorization support: reference F32 has aligned access   [ central2d.h(411,17) ]
         remark #15388: vectorization support: reference F32 has aligned access   [ central2d.h(411,17) ]
         remark #15388: vectorization support: reference F32 has aligned access   [ central2d.h(411,17) ]
         remark #15388: vectorization support: reference F32 has aligned access   [ central2d.h(411,17) ]
         remark #15388: vectorization support: reference F32 has aligned access   [ central2d.h(411,17) ]
         remark #15388: vectorization support: reference F32 has aligned access   [ central2d.h(411,17) ]
         remark #15388: vectorization support: reference F32 has aligned access   [ central2d.h(411,17) ]
         remark #15388: vectorization support: reference F32 has aligned access   [ central2d.h(411,17) ]
         remark #15388: vectorization support: reference F32 has aligned access   [ central2d.h(411,17) ]
         remark #15388: vectorization support: reference F32 has aligned access   [ central2d.h(411,17) ]
         remark #15388: vectorization support: reference F32 has aligned access   [ central2d.h(411,17) ]
         remark #15388: vectorization support: reference F32 has aligned access   [ central2d.h(411,17) ]
         remark #15427: loop was completely unrolled
         remark #15417: vectorization support: number of FP up converts: single precision to double precision 4   [ central2d.h(411,17) ]
         remark #15418: vectorization support: number of FP down converts: double precision to single precision 1   [ central2d.h(411,17) ]
         remark #15301: SIMD LOOP WAS VECTORIZED
         remark #15448: unmasked aligned unit stride loads: 20 
         remark #15449: unmasked aligned unit stride stores: 1 
         remark #15475: --- begin vector loop cost summary ---
         remark #15476: scalar loop cost: 96 
         remark #15477: vector loop cost: 13.500 
         remark #15478: estimated potential speedup: 6.850 
         remark #15479: lightweight vector operations: 54 
         remark #15487: type converts: 5 
         remark #15488: --- end vector loop cost summary ---
      LOOP END
   LOOP END
LOOP END
\end{lstlisting}
}

The last stage in this phase, which we are uncertain if it is working correclty, is to create a custom allocator fot \texttt{std::vector}.  Unfortunately there is no byte aligned allocator in the standard yet, so we snagged somebody elses that seems to be popular on the interweb:\\\\ \centerline{https://gist.github.com/donny-dont/1471329}\\

We believe by virtue of the fact that it gives the right simulation that we are at least have a valid allocator, but honestly at this point we are very frustrated with how difficult it has been to use \texttt{std::vector}.  After declaring

{\tiny
\begin{lstlisting}
    #define BYTE_ALIGN 64
    #ifdef __INTEL_COMPILER
        typedef __declspec(align(BYTE_ALIGN)) std::vector<vec, aligned_allocator<vec, BYTE_ALIGN>> aligned_vector;
    #else // GCC
        typedef __attribute__ ((aligned(BYTE_ALIGN))) std::vector<vec, aligned_allocator<vec, BYTE_ALIGN>> aligned_vector;
    #endif
    aligned_vector u_;            // Solution values
    aligned_vector f_;            // Fluxes in x
    aligned_vector g_;            // Fluxes in y
    aligned_vector ux_;           // x differences of u
    aligned_vector uy_;           // y differences of u
    aligned_vector fx_;           // x differences of f
    aligned_vector gy_;           // y differences of g
    aligned_vector v_;            // Solution values at next step   
\end{lstlisting}
}

\noindent it turns out that memory aligned allocators (in general) greatly conflict with other locations in the code that were previously vectorized.  So although we were able to get some interesting vectorization results from the above, we may switch over to C-style arrays in the near future.  The chaos of the above code, in particular the newly vectorized loop, do not seem worth the trade-off of how ugly / difficult to follow it is.

We may also explore other allocators e.g. using Eigen's aligned allocator, but generally do not see the benefit as we have probably spent more time trying to utilize these vectors through arcane trickery than we would have to just rewrite it based off of say \texttt{float *u} and use a memory aligned malloc.  Time will tell whether we decide to keep these or not.

\subsubsection{Manual Vectorization}
\label{sec-tune-vectorizing-manual-vectorization}

When it comes to writing custom kernels, we actually are very excited to do this.  After much deliberation we have finally been able to get \texttt{\#pragma offload target(mic)} to cooperate with things like \texttt{std::vector} and \texttt{std::array}.

Getting them working on the Phi's was as far as we got at this point, as discussed in section~\ref{sec-parallel-device}, so we hope to be able to play with this more and compare with the intel compiler's vectorization and \emph{maybe} even beat it!  If anything, we can just pull the same trickery we did in the corrector step described in the previous section.

% \subsection{Eliminating Redundant Computation}
% \label{sec-tune-eliminating}

% \subsection{Offloading vs. Natively Executing Parallel Kernel}
% \label{sec-tune-offloading}

% \subsection{Overlapping Computation with Communication}
% \label{sec-tune-overlapping}

% \subsection{Implementing Alternative Physics Modeling Algorithms}
% \label{sec-tune-implementing}
